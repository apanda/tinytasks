\begin{abstract}
Operating systems and network designs have shown that small units of work
improve utilization and responsiveness. This paper argues for a similar model to
be used in datacenters by using \textit{tiny tasks} that complete in hundreds of
milliseconds.  Tiny tasks alleviate the long wait times for interactive,
user-facing jobs seen in today's clusters and avoid the need for complex skew
mitigation techniques by evenly spreading work across available resources. 
In current data-parallel computing frameworks, high task launch overheads, lack
of scalable file systems, schedulers prevent users from running short
tasks. While, recent improvements have addressed some of these bottlenecks,
there are numerous challenges in converting \emph{all} jobs into tiny tasks. We
discuss the design goals for task execution that can support tiny tasks and
present a preliminary architecture to realize this goal.
\end{abstract}

