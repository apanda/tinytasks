\begin{abstract}
In data-parallel computing frameworks, certain engineering limitations such as scalability
of the name node and scheduling dictates that users should use tasks with long task runtimes.
However, long-running tasks result in
long wait-times for interactive, user-facing jobs in mixed use clusters.
Furthermore, despite the use of elaborate skew mitigation
techniques, system designers continue to struggle with more evenly distributing
the work required for a job across tasks.
Recent improvements in distributed file systems and scheduling have eliminated
many of the reasons why long task runtimes have been historically preferred.
In this paper we argue for a move to \emph{tiny tasks},
tasks with runtimes of a few $100$ milliseconds.
Tiny tasks enable dynamic repartitioning of work in a job, and thus minimizing
the necessity for pre-partitioning work across tasks.
Furthermore, for mixed workloads, the use of tiny tasks limits the time an
interactive job must wait before its tasks are run, allowing for  increased
cluster utilization, and improving responsiveness, and job fairness.
We quantify these simulations, and show that smaller tasks can improve
response time by a factor of \fixme{5}. We also discuss some challenges
faced in building a system using tiny tasks.
\end{abstract}
