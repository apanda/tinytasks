\begin{abstract}
% Colin's shot:
% Breaking large units of work into smaller tasks is a well-known technique in operating systems and networking for improving responsiveness and utilization.
%Operating systems and network designs break large units of work into
%smaller tasks in order to improve responsiveness and utilization.
%Operating systems and network designs use small units of work
%to improve utilization and responsiveness.
We argue for breaking data-parallel jobs
into \textit{tiny tasks} that each complete in hundreds of milliseconds.
Tiny tasks avoid the need for complex skew mitigation techniques: by
breaking a large job into millions of tiny tasks, work will automatically
be spread evenly over available resources by the scheduler.
Furthermore, tiny tasks alleviate the long wait times seen in today's
clusters for interactive jobs, since even tasks for batch jobs
complete quickly.
% Maybe remove this sentence?
% Thus, tiny tasks allow for increased utilization without sacrificing
% responsiveness or fairness.
We demonstrate that small tasks can improve response times by a factor of
8.

In current data-parallel computing frameworks, high task launch
overheads and scalability limitations prevent users from running short tasks.
Recent research has addressed many of these bottlenecks; this work
discusses remaining challenges and proposes a task execution
framework that can support tiny tasks.
%While recent research has addressed some of these bottlenecks, converting
%\emph{all} jobs into tiny tasks requires addressing numerous other challenges.
%We discuss the design goals for a task execution framework that can support
%tiny tasks, and present a preliminary architecture to realize this goal.

\eat{

This paper argues for a similar model to
be used in datacenters by using \textit{tiny tasks} that complete in hundreds of
milliseconds.  Tiny tasks alleviate the long wait times for interactive,
user-facing jobs seen in today's clusters and avoid the need for complex skew
mitigation techniques by evenly spreading work across available resources. 
In current data-parallel computing frameworks, high task launch overheads, lack
of scalable file systems, schedulers prevent users from running short
tasks. While, recent improvements have addressed some of these bottlenecks,
there are numerous challenges in converting \emph{all} jobs into tiny tasks. We
discuss the design goals for task execution that can support tiny tasks and
present a preliminary architecture to realize this goal.
}
\end{abstract}

