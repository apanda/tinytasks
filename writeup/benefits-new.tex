\section{Benefits of Tiny Tasks}
Tiny tasks benefit data center workloads by providing inherent elasticity:
fine-grained units of work can be dynamically allocated to machines as
resources become available, eliminating stragglers and issues of data skew,
and allow jobs to use available resources without sacrificing fairness
when new jobs are submitted.

\subsection{Inherent handling of skew and stragglers}
\begin{enumerate}
\item Sources of data skew: key skew, value skew, ...
\item Ideal improvement (show CDF of max/median job time, reduce skew maybe, and Panda's binpacking graph)
\item Experimental results: improvements are possible simply by using this
design principle in today's frameworks! Josh's results here
\end{enumerate}
This is Josh's stuff (should probably dump the crap simulation I did for this)

\subsection{Fairness without Sacrificing Utilization}
Diagram here showing a before and after of what happens when a new job arrives
(and another job was already using the entire cluster)

\subsection{Seamlessly Integrating Batch and Interactive Workloads}
should this be merged with the above?
\begin{enumerate}
\item Kay's simulation of response times for differently sized jobs as
parallelization increases
\end{enumerate}

