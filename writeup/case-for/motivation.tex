\section{Trends \& Motivation}

Common problems seen in datacenters. Probably need a graph for each one of these
\begin{myitemize}
  \item  Some tasks are very short/small and others are very large (in terms of
    resource usage) and long (in terms of time) \todo{Graph}.  
    Scheduling these is difficult; 

  \item Stragglers are exacerbated by the above problem~\todo{Graph}: if a task
    ends up on a machine with a large task consuming tons-o-resources, it will
    straggle

  \item Hot spots: there are hot spots in network usage, and also on machines
    that host popular data
\end{myitemize}

Make a table of problems caused by power-law distributions and papers about
them (should make this more concrete/compelling)

Datacenters and workloads are changing
\begin{myitemize}
  \item Focus is on main memory workloads, with need for low latency. Use the
    MapReduce$\rightarrow$Dremel$\rightarrow$Spark Streaming data here.
  \item Networks have grown to be fast making disk locality irrelevant. This
    means getting data from a remote location is fast. Further networks don't
    have a fixed overhead of transfer. Disk-based stuff uses 64MB to amortize
    that cost. \todo{Numbers ?} \panda{I am not quite sure why amortization and
    networks matter, in some sense if you access disks you should need the same
    amortization, which is under 64 M. On the other hand it is easier to move tasks
    around}
  \item SSDs are becoming cheaper - Fast random access with low access time.
    With growing memory sizes + SSD, it is feasible to fetch most of the data
    without hitting the disk (local or remote).
  \item Need for low latency and ability to exploit fast-data access leads to
    tiny tasks
  \item Workloads have become more heterogeneous, with batch and interactive jobs
  showing up in the same cluster more often, and also with a range of job sizes.
\end{myitemize}
