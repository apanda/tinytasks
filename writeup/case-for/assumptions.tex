\section{Assumptions}
Our analysis targets large scale clusters running a mixture of analytical and interactive queries of the kind that are
prevalent today. While one could achieve results similar to ours through other means, we assume that cluster computing
frameworks have control over tasks, which are atomic units of work. A cluster compute framework can only schedule, or
kill processes at the granularity of a task, and either all of a task completes, and its output is used by other tasks
in the job, or none of a task finishes. Furthermore, we assume that a task cannot be preempted, it can only be killed.

We also assume, in a manner similar to what is available today, machines in a cluster have a certain number of
processing slots, and a certain amount of memory, that the cluster scheduler can allocate such memory and CPU, and can
get reasonable isolation from other tasks that are run on that machine. 

Finally our last assumption is that the network is reasonably fast, and full-bisection, such that the placement of tasks
is no longer a large concern. This is especially important as it allows for better load balancing, since tasks can be
placed on another machine more easily.
