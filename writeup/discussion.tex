\section{Alternate Designs and Related Work}
\label{sec:alternate}

Tiny tasks solve two major problems: outliers, and sharing
a cluster between batch and interactive or user-facing jobs. A variety of
approaches solve one of the two problems in isolation; e.g., skew handling
techniques mitigate outliers, and process migration allows improved
sharing between long and short jobs.  Unlike these approaches, tiny tasks
provides a simple design paradigm that solves both problems.

\subsection{Preemption and Process Migration}
%\subsection{Alternate Design Schemes}

Our choice of a cooperative multi-tasking scheme with tiny tasks 
is in contrast to preemption based schemes with long running tasks commonly used 
in operating systems. The benefits of process preemption are well studied in operating
systems literature; for instance~\cite{sherman1972trace} shows the benefits
of short scheduling quanta. Work in distributed
operating systems~\cite{douglis1991transparent,milojivcic2000process,rozier1991overview} and virtual machines~\cite{tanenbaum1990experiences}
has applied preemption in a distributed context by exploring migrating
processes between machines.
Compared to tiny tasks, one advantage of preemption is that scheduling quanta
can be tightly controlled, since the framework (or operating system) controls
when the task is preempted. However, preemption has other disadvantages:

%However achieveing low overhead task-switching is more expensive in the distributed setting.
\vspace{4pt}\noindent\textbf{Cost of task-switching:}
Work on process
migration~\cite{douglis1991transparent,milojivcic2000process} and virtual
machine migration~\cite{clark2005live} has shown mechanisms to transparently
move tasks across machines. Such methods, however, involve a high overhead, as
migrating a task involves transferring both task context, a task's intermediate
data, and its inputs. For data parallel applications, input data, and intermediate
data can be several gigabytes, incurring very high overheads.

%Cooperative multitasking, by contrast, eliminates the need to migrate such data. In
%addition, we envision moving shared state into a framework-maintained scratch space.
%By explicitly differentiating such state, we envision that it can be more easily migrated.

%tiny tasks represents the unit of recovery as they implement deterministic
%operations (need to say this explicitly). In a preemptive model we need the
%system to checkpoint and log execution in order to be able to transparently
%recover. This has a high overhead, in particular, in the case of intensive data
%apps. Cite some VM papers here… 

\vspace{4pt}\noindent\textbf{Fault tolerance:}
In a data center setting, tiny tasks are better suited for
fault tolerance when compared to task preemption. Since tiny tasks
are deterministic, each task is unit of work that can be executed in parallel
 during recovery~\cite{zaharia2012discretized}. In contrast, while using 
preemption with larger tasks, the system will need to perform expensive
operations like logging task execution events to replay~\cite{dunlap2002revirt}
and fault recovery cannot be executed in parallel.

Datacenter schedulers like Quincy~\cite{isard2009quincy} kill tasks on
preemption and trade-off between responsiveness and wasting the work of tasks
that are preempted.  Recent work proposed Amoeba~\cite{ananthanarayanan2012true},
a system that uses preemption in the context of MapReduce-like
cluster computing frameworks. Amoeba identifies safe-points where a task can be
paused and restarted elsewhere without wasted work. The main drawback of Amoeba
is that it does not provide a mechanism for determining such safe-points, which
is difficult for general tasks (even tasks that use the MapReduce programming
model).  The Amoeba authors choose preemptability rather than small tasks for
two reasons. First, they cite high task launch overheads in systems like
Hadoop; as described in~\S\ref{sec:prog}, these overheads are not fundamental
and can be solved with improved engineering. Second, they note that creating
\emph{uniformly} sized small tasks is difficult. Tasks need not be uniformly
sized for the benefits of tiny tasks to hold; rather, tasks must be orders of
magnitude smaller than today's tasks.

\subsection{Coarse-grained Partitioning}
Omega, Google's newest cluster scheduler~\cite{melnik2010dremel},
was designed to share cluster
resources between batch and interactive workloads. However, Omega achieves
this by coarse-grained partitioning of cluster resources.
Interactive services like Dremel~\cite{melnik2010dremel} spin up long-running
agents that serve incoming queries, rather than scheduling new resources for
each request.  Such coarse-grained partitioning limits utilization because each service
must be provisioned to handle peak load, and one service's extra capacity
cannot easily be reallocated to another service.

\subsection{Skew-Handling}
A separate line of research has focused on skew-mitigation to improve job
performance in data centers. Examples of such work include
Mantri~\cite{ananthanarayanan2010reining}, SkewTune~\cite{kwon2012skewtune},
Scarlett~\cite{ananthanarayanan2011scarlett}, and work on task
speculation~\cite{zaharia2008improving}. Mantri and Scarlett attempt to
mitigate task runtime skew by modeling the causes for skew and accounting for
these causes when scheduling tasks. In particular, Mantri performs resource aware scheduling to decrease the
probability of observing task skews, while Scarlett replicates storage blocks
based on their probability to decrease the wait time for a popular block. While
both of these systems moderately reduce task skew, they rely on a fragile set of
signals, and do not work in all cases.
%Using tiny tasks naturally overcomes data
%skew among reduce tasks as fine-grained hash partitioning ensures that the data is
%more evenly spread across tasks. As shown in
%Section~\ref{sec:benefits}, using tiny tasks also allows work to be balanced across
%different machines thereby overcoming skew due to slow machines.
Furthermore, existing skew mitigation techniques trade-off cluster resources to
gain more predictable task runtimes. This limits the applicability of these
techniques under situations of high load, where such guarantees might be most
important.

\subsection{Fine-grained Sharing}
The idea of using smaller units of work to improve load balancing is well
studied.  In multi-threaded applications, work-stealing~\cite{blumofe1994scheduling}
based schedulers divide work at very fine granularities to provide better load
balancing.  Smaller units of work have also been used in many other contexts
including operating systems~\cite{sherman1972trace}, storage systems~\cite{ghemawat2003google,
chang2008bigtable} and distributed hash tables~\cite{stoica2001chord}. We apply this
principle to the granularity of tasks scheduled in large-scale clusters.  The
idea of sharing cluster resources at task-level granularity has been used in
existing cluster schedulers~\cite{hindman2011mesos, zaharia2010delay} and prior
proposals have also looked at splitting MapReduce tasks~\cite{bhatotia2011incoop} to
address some of the problems.  Using tiny tasks makes sharing cluster resources
more effective and enables better responsiveness and fairness for users.

%  Fair sharing 
% of cluster resources across multiple users has bee proposed in 

% Mesos~\cite{hindman2010mesos}
