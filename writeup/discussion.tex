\section{Alternate Designs and Related Work}
\label{sec:alternate}

Tiny tasks solve two major problems: stragglers, and sharing
a cluster between batch and interactive jobs. A variety of
approaches solve one of the two problems in isolation; e.g., skew handling
techniques mitigate stragglers, and process migration allows improved
sharing between long and short jobs.
%Unlike these approaches, tiny tasks
%provides a simple technique to solve both problems.

\subsection{Preemption and Process Migration}
\label{sec:preemption}
%\subsection{Alternate Design Schemes}

Our choice of a cooperative multi-tasking scheme with tiny tasks 
contrasts preemption based schemes commonly used 
in operating systems.
%The benefits of using process preemption to enforce
%short scheduling quanta are well studied in operating
%systems literature (e.g., ~\cite{sherman1972trace}).
%demonstrates the benefits of
%using preemption to enforce short scheduling quanta.
%Work in distributed
%operating systems~\cite{douglis1991transparent,milojivcic2000process,rozier1991overview} and virtual machines~\cite{tanenbaum1990experiences}
%has applied preemption in a distributed context.
% by migrating
%processes between machines.
Compared to tiny tasks, one advantage of using preemption to guarantee sharing
properties is that the system
can tightly control scheduling quanta~\cite{sherman1972trace,tanenbaum1990experiences};
however, preemption has other disadvantages:

%However achieveing low overhead task-switching is more expensive in the distributed setting.
\vspace{4pt}\noindent\textbf{Cost of task-switching:}
Prior work has proposed mechanisms to migrate
processes~\cite{douglis1991transparent,milojivcic2000process}, virtual
machines~\cite{clark2005live}, and services~\cite{rozier1991overview} across
machines.
%Work on process
%migration~\cite{douglis1991transparent,milojivcic2000process} and virtual
%machine migration~\cite{clark2005live} proposes mechanisms to transparently
%move tasks across machines.
%Such methods involve a high overhead, as
Migrating tasks involves transferring inputs, context, and intermediate
data; for data parallel applications, input data and intermediate
data can be several gigabytes, incurring high migration overhead.

%Cooperative multitasking, by contrast, eliminates the need to migrate such data. In
%addition, we envision moving shared state into a framework-maintained scratch space.
%By explicitly differentiating such state, we envision that it can be more easily migrated.

%tiny tasks represents the unit of recovery as they implement deterministic
%operations (need to say this explicitly). In a preemptive model we need the
%system to checkpoint and log execution in order to be able to transparently
%recover. This has a high overhead, in particular, in the case of intensive data
%apps. Cite some VM papers here… 

\vspace{4pt}\noindent\textbf{Fault tolerance:}
If a long task fails, it must be re-executed from the beginning. Periodic task
checkpointing
can speed recovery, but at great expense~\cite{dunlap2002revirt}.
On the other hand, the short duration of
tiny tasks limits the amount of lost work after a failure, and tiny
tasks can be executed in parallel to speed recovery.
%tiny tasks
%represent small units of work that can be executed in parallel during recovery.
%each task is a unit of work that can be executed in parallel during recovery.
%Furthermore, unlike with tiny tasks,
%fault recovery cannot be parallelized.
%Tiny tasks are better suited for
%fault tolerance than task preemption. Since tiny tasks
%are deterministic, each task is unit of work that can be executed in parallel
% during recovery~\cite{zaharia2012discretized}.
%In contrast, while using 
%preemption with larger tasks, the system will need to perform expensive
%operations like checkpointing task execution events~\cite{dunlap2002revirt},
%and fault recovery cannot be executed in parallel.

In spite of these drawbacks, some schedulers for data parallel application
use preemption.
Quincy~\cite{isard2009quincy} kills tasks on
preemption, trading wasted work for responsiveness.
%ting the work of tasks
%that are preempted.
Amoeba~\cite{ananthanarayanan2012true} uses preemption to provide improved elasticity in the context of 
MapReduce-like cluster computing frameworks. 
%Recent work proposed Amoeba~\cite{ananthanarayanan2012true},
%a system that uses preemption in the context of MapReduce-like
%cluster computing frameworks.
Amoeba identifies safe-points when a task can be
paused and restarted elsewhere without wasted work. The main drawback of Amoeba
is that it does not provide a mechanism for determining safe-points, which
is difficult for general tasks (even tasks that use the MapReduce programming
model).  The Amoeba authors choose preemptability rather than small tasks for
two reasons. First, they cite high task launch overheads in systems like
Hadoop; as described in~\S\ref{sec:prog}, these overheads are not fundamental
and can be solved with improved engineering. Second, they note that creating
\emph{uniformly} sized small tasks is difficult. Tasks need not be uniformly
sized for the benefits of tiny tasks to hold; rather, tasks must be orders of
magnitude smaller than today's tasks.

\subsection{Coarse-Grained Partitioning}
Many clusters use static partitioning to ensure that long-running jobs do not
cause head of line blocking for newly arriving, high priority
jobs~\cite{thusoo2010data}. Static partitioning limits utilization because each
partition must be provisioned to handle peak load, and extra capacity cannot
easily be reallocated.


Omega~\cite{wilkes2013omega}, Google's cluster scheduler, uses more flexible coarse
grained partitioning. Omega shares a cluster across many frameworks that
each perform their own task-level scheduling. Tiny tasks can improve performance
for a single framework in this context, but performing task-level scheduling
separately for each framework limits cluster utilization.
Tiny tasks reap the greatest efficiency benefits when used to share resources
between multiple users and frameworks.
%such systems limit utilization by
%performing task-level scheduling separately for each framework. Tiny tasks reap
%the greatest benefit when used to share resources between different frameworks at
%task-level granularity.
\eat{
Omega, Google's cluster scheduler~\cite{wilkes2013omega},
shares cluster resources between batch and interactive workloads using
coarse-grained partitioning of cluster resources.
Interactive services like Dremel~\cite{melnik2010dremel} spin up long-running
agents that serve incoming queries, rather than scheduling new resources for
each request.  
Such coarse-grained partitioning limits utilization because each service
must be provisioned to handle peak load, and one service's extra capacity
cannot easily be reallocated to another service.
}
\subsection{Fine-grained Sharing}
The idea of using smaller units of work to improve load balancing is well
studied.  In multi-threaded applications, work-stealing
based schedulers~\cite{blumofe1994scheduling} divide work at very fine granularities to provide better load
balancing.  Smaller units of work have also been used in
operating systems~\cite{sherman1972trace}, storage systems~\cite{chang2008bigtable,ghemawat2003google} and distributed hash tables~\cite{stoica2001chord}. We apply this
principle to tasks scheduled in large-scale clusters.  The
idea of sharing cluster resources at task-level granularity has been used in
existing cluster schedulers~\cite{hindman2011mesos, zaharia2010delay} and prior
proposals have also looked at splitting MapReduce tasks~\cite{bhatotia2011incoop} to
address some of the problems.
Tiny tasks drive the idea of small tasks to the extreme to enable improved
sharing of cluster resources
and better responsiveness.

\subsection{Skew-Handling}
A separate line of research has focused on skew-mitigation to
mitigate the effect of stragglers.
%improve job performance in data centers.
Examples of such work include
Mantri~\cite{ananthanarayanan2010reining}, SkewTune~\cite{kwon2012skewtune},
Scarlett~\cite{ananthanarayanan2011scarlett}, work on task
speculation~\cite{zaharia2008improving}, and work on task cloning~\cite{ananthanarayanan2013effective,dean2013the}. Mantri and Scarlett attempt to
mitigate task runtime skew by modeling the causes for skew and accounting for
these causes when scheduling tasks. In particular, Mantri performs resource aware scheduling to decrease the
probability of observing task skews, while Scarlett replicates storage blocks
based on their probability to decrease the wait time for a popular block. Both of these systems moderately reduce task skew, but they rely on a fragile set of
signals and do not work in all cases.
%Using tiny tasks naturally overcomes data
%skew among reduce tasks as fine-grained hash partitioning ensures that the data is
%more evenly spread across tasks. As shown in
%Section~\ref{sec:benefits}, using tiny tasks also allows work to be balanced across
%different machines thereby overcoming skew due to slow machines.
Furthermore, existing skew mitigation techniques use additional cluster resources to
gain more predictable task runtimes. This limits the applicability of these
techniques under situations of high load, when achieving predictability may be
most important.
%re such guarantees might be most
%important.


