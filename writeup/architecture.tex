\section{Architecting for Tiny Tasks}
% TODO(josh): maybe the bit about "different kinds" can be made clearer?
Designing a cluster framework for tiny-tasks requires addressing multiple
architectural challenges. We argue for a scheduler that can handle hundreds of
thousands of scheduling decisions per second, and schedule tasks with
fewer than $100 \mu$s of overhead.  Supporting tiny tasks also requires a
scalable filesystem that can handle on-disk data blocks of $8$MB or less.
Finally, a framework for tiny tasks must provide a programming model that
supports a wide variety of jobs.

\subsection{Execution Model}
We propose an execution model similar to today's cluster frameworks (e.g.,
MapReduce~\cite{??}, Spark~\cite{??}). Each job is composed of a large number
of tasks that each represent an indivisible and idempotent unit of
execution. A task is comprised of a set of named
inputs and code that operates on these inputs, and each task runs on a single
machine.
%In our
%model, inputs for a task are small, usually $8$MB or smaller, and tasks execute in
%around $100$ ms. 
% Kay: isn't the following true with MR today?
%Our execution model also requires that the framework, rather than the
%task be responsible for fetching inputs from disk, and writing outputs to disk, or transmitting
%them over the network. This allows the framework to prefetch inputs for tasks, and
%to optimize whether outputs are written to disk or kept in memory. We envision using
%a cooperative scheduling model, i.e., tasks must explicitly hand back resources (by finishing)
%before these resources are reassigned. Task runtimes are limited by limiting input size.
%Section~\ref{sec:alternate} compares our design to other alternates.

\subsection{Low-latency scheduling}
Supporting tiny tasks requires a low-latency, high throughput cluster scheduler.
In a cluster with a $10,000$ slots (e.g., 1250 8-core machines),
supporting $100$ms tasks implies that the scheduler must make $10^5$ scheduling
decisions per second, on average.
While today's centralized schedulers have well-known scalability
limits~\cite{wilkesberkeley} that
would prevent them from supporting tiny tasks in a large cluster,
recently proposed distributed scheduling techniques like batch
sampling~\cite{ousterhoutbatch} provide a near-optimal and far more scalable
alterative to centralized techniques.
Our proposed system relies on the use of such distributed schedulers.
%near optimal scheduling, while also exhibiting better scalability. Our proposed system relies
%on the use such distributed schedulers, and extensions supporting fairness guarantees, utilization
%guarantees, and other such properties. To reduce additional task latency due to data movement, our
%proposed scheduler will also support task constraints, and it has already been demonstrated that
%these constraints can be enforced by techniques like batch sampling. 

In addition to providing high throughput scheduling decisions, a framework for
tiny tasks must also reduce the overhead
for launching individual tasks. Existing frameworks like Hadoop MapReduce
have task launch overheads of many seconds, because they launch a new Java
Virtual Machine (JVM) for each task. Newer frameworks, e.g., Spark, reduce task
launch overhead to tens of milliseconds by instead running tasks in an
existing JVM.  In order to support sub-second tasks while also ensuring that
task launch overhead is not a significant fraction of task runtime, we propose a system
that provides task launch overheads of $100 \mu$s. While significantly
lower than overheads provided in current systems, we believe that this low
overhead is attainable in a well-engineered system:
studies of network round
trip times in datacenters have demonstrated that round trip times of 10$\mu$s
are attainable today, and launching a task in Java by simply
spawning a new thread requires less than $1\mu$s

\subsection{Pipelined Execution}
\label{sec:pipeline}
To optimize for short tasks, we propose pipelining the phases of task execution.
Current task execution generally happens in three phases: the fetch phase fetches inputs from a file or over the network; the compute phase executes the code
for the task; and the output phase stores the outputs in memory or on disk.
We propose pipelining these phases such that one task's input is fetched while
another task is running, thus more fully utilizing both the CPU and the I/O
bus.

Pipelining is particularly powerful when combined with small tasks. With
today's large tasks, tasks accumulate large amount of output data in memory.
Often, a task's output data will overflow the available space in memory
and spill onto a disk, which is a well-known cause of poor MapReduce
performance~\cite{toddlipconthing}. Tiny tasks fundamentally change the
resource footprint of tasks: since tasks run for a shorter period of time,
they generate less output data and thus use less memory. The framework can
expicitly control the remaining memory, choosing to cache the most important
and most frequently-accessed output data, and store the remaining data on disk
or on a different machine. Data written to disk can be written in the
background, while more tasks for the job are run, alleviating the performance
problems from stalling a task to overflow in-memory data to disk.

Pipelining shifts control of I/O entirely to the framework, allowing for
powerful optimizations.  For a MapReduce-style job, for instance, the
framework could store the map outputs that will be used for the first set of
tiny reduce tasks in memory, and store the remaining outputs on disk. While
the first set of tiny reduce tasks are running, the framework can pipeline
reading data for the next set of tasks.  Giving frameworks better control over
I/O has previously been shown~\cite{chowdhury2011managing, cohwdhury2012coflow}
to greatly improve network utilization and performance in clusters.

\subsection{Scalable Storage Systems}
To allow tasks to complete quickly, we propose operating on data blocks of
at most $8$MB. Even with pipelining, we
expect a task's inputs to be readable within the runtime of a single task. Current disks exhibit
throughputs of approximately $100$MB/s, and thus a single tasks inputs can be no more than $10$MB.
Based
on these calculations, and results from FDS~\cite{nightingale2012flat} we require using data blocks
that are no more than $8$MB. Reducing blocks to this size negatively impacts the performance for
existing distributed filesystems like HDFS which use a centralized metadata server. Recent work on
distributed filesystems like FDS address these concerns by distributing metadata cross multiple
servers. While FDS provides the first steps towards a filesystem for tiny tasks, we envision
one would need to make changes to more fully integrate this system with the scheduler, and
the pipelining component.

\subsection{Programming model}
\label{sec:prog}
While splitting most tasks in a dataparallel framework by reducing input size 
is relatively straight forward, splitting certain tasks might incur a higher overhead 
than the corresponding improvements. For instance, consider an all-to-all shuffle, involves
communication across $O(N^2)$ pairs of senders and receivers. The increased communication cost
for such tasks might not be offset by the additional degree of parallelism. Aggregation trees are
commonly used in data parallel frameworks like Dryad~\cite{yu2008dryadlinq} to solve similar
problems by progressively combining outputs from multiple tasks. We plan on using such trees,
and extensions to these techniques to provide tools that can allow frameworks to reduce task
lengths for arbitrary tasks.

There might however still remain tasks, which are either not associative and commutative, or
require temporary state, which might not be easily divided up. We propose providing a 
framework maintained temporary store that can be used by tasks to communicate, and reuse data
across invocations. Such a store could for instance be used to implement a task computing distinct
values for a file. We envision that such a store would provide an interface similar to key-value
stores, and would provide strong consistency guarantees.

Finally there are those tasks which cannot be split despite these tools. To accommodate these 
tasks we plan on allowing some large tasks to run on the cluster. To provide the guarantees offered
by tiny tasks, such tasks would be required to run on a small part of the cluster reserved for large
tasks. Larger reservations allow such tasks to finish quicker, while potentially reducing cluster 
utilization. We envision providing the cluster administrator with controls to change the precise mixture.
Exploring the effects of such sharing, and the tradeoff space offered by such control is subject of ongoing research.
