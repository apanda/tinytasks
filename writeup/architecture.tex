\section{Architecting for Tiny Tasks}
Tiny tasks present many challenges in designing a computing framework.  Breaking
down jobs in clusters into tiny execution units implies the need for
low-overhead high-throughput scheduling, scalable storage systems and a
programming model that can be used to support different kinds of computations.
In this section, we present our design goals and outline the challenges in
realizing them.

\subsection{Execution Model}
We propose an execution model where every job in the cluster is split into a
large number of tiny tasks. Similar to existing frameworks, each task is a
deterministic unit of execution and a task's description contains the code to be
executed and a set of named inputs chunks. We enforce that each input chunk is
restricted in size to 8MB and that tasks execute in around $100$ms.  Unlike
existing systems, our task execution framework pre-fetches a task's input into
memory(~\ref{sec:pipeline}) before running it. Tasks produce a set of named
outputs that can be used as inputs by subsequent tasks. We use a co-operative
multi-tasking scheme where the scheduler waits for a task to finish before launching
the next task.  Task running times are controlled by using small input blocks
and support from programming frameworks. We compare against alternate design
schemes in Section~\ref{sec:alternate}.

\subsection{Low-latency scheduling}
In a datacenter containing thousands of machines, having tasks which execute for
$100$ms would mean that cluster schedulers need to make millions of scheduling
decisions every second. Centralized scheduling schemes have well known
scalability limits~\cite{john-wilkes-berkeley} and distributed scheduling
schemes will be required to handle tiny tasks. Recently proposed distributed
scheduling techniques like batch sampling~\cite{sparrow} have been shown to 
provide near-optimal scheduling decisions. We plan to explore distributed 
scheduling schemes which can also provide fairness guarantees and balance
utilization of machines across the cluster. To minimize data-movement
distributed scheduling schemes will also have to consider locality constraints
and techniques like delay-scheduling~\cite{delay} can be used to maximize locality at the
cost of latency. 

In addition to scheduling latency, task launch overheads will have to be
minimized to ensure that tasks can spend a large fraction of their execution
time doing useful work. Existing frameworks like Hadoop typically launch a new
JVM for each task and launching new tasks takes \fixme{5-10} seconds. Newer
frameworks like Spark reuse existing JVMs, but task launch overhead is still on
the order of a few milliseconds~\cite{sparrow}.

We believe that average task launch overhead can be reduced $100 \mu$s making it
an insignificant portion of the execution time. Much of the existing task launch
overhead stems from shipping the task over the network and this can be
eliminated by caching tasks. Task launch latency can then be reduced to one
intra-datacenter RPC, containing a checksum of the task description, path to
task inputs, and the time required to execute a new thread. Furthermore, 
scheduling decisions where a task continues to execute in a given slot can be 
efficiently implemented using a lightweight renewal message similar to leases.\\

\subsection{Pipelined Execution}
\label{sec:pipeline}
Task execution can be made more efficient by using a pipelined execution scheme.
Every tiny task can be broken down in to three main phases: a fetch phase where
the task descriptions and the inputs are fetched consuming I/O resources, an
execute phase where task consumes CPU and an output phase where the tasks
outputs are saved to the storage system. By pipelining CPU and I/O resources
across different tasks, we can further reduce associated overheads. With tiny
tasks we expect task inputs and outputs to be small as well and this allows the
execution framework to cache these values and flexibly schedule data transfers.
Using a data transfer interface similar to coflows~\cite{coflow-hotnets} we
believe that explicitly managing data transfer between tasks will allow us to
optimize network utilization in the data center.

\subsection{Scalable Storage Systems}
In order for tasks to be short, they must operate on a small amount of on-disk
data, or use in-memory data.  For data stored on-disk, assuming data can be read
from disk at 100MB/s, a task can read no more than 10MB of data in 100ms.  Hence
we propose that blocks in the distributed file system must be limited to around
8MB. Existing file systems like HDFS, which use a centralized metadata server,
hit scalabilty limits with smaller block sizes~\cite{facebook-berkeley-talk}.
Recent work on Flat Data Center Storage~\cite{fds}, addresses this problem by
distributing metadata across servers and their results indicate that tasks can
overcome the need for disk locality. While FDS provides a natural environment
for tiny tasks, we believe there is a need to consider in-memory locality as
well. Accessing in memory data~\cite{pacman, spark} will enable tasks to access 
up to 1GB in 100ms and an integrated approach to caching, scheduling based on
memory locality and appropriate placement on disks will be necessary to handle
tiny tasks.

\subsection{Programming model}
\label{sec:prog}
Splitting jobs into tiny tasks is relatively straight forward for data parallel
programs as programming frameworks can sub-divide the input to many more tasks.
However some computations could become more expensive as we increase the total
number of tasks. An example of this would be MapReduce computations where we
have an all-to-all shuffle ($O(N^2)$ in communication) and benefits from higher
degree of parallelism may not enough to offset the costs. Combiners, which can
perform partial aggregation of data, are commonly used in computations which are
associative and commutative. Extending this, we plan to explore using
aggregation trees~\cite{something} which can address this problem and ensure
that tasks remain tiny.

However we could still have computations that are either not associative,
commutative or require a large amount of temporary state (e.g. computing
distinct elements). To handle such computations, we propose providing 
support for a distributed ``scratch space'' that can be used by tasks to 
communicate across different executions. For example, to compute the set of
unique elements in a file, tasks could use the scratch space to store elements
seen so far. We envision that a simple key-value store like interface would
be sufficient for this purpose and plan to provide strong 
consistency~\cite{something} for accesses within a datacenter. 

As an exception, we also plan to accomodate some large tasks in the cluster.
This would require programmer intervention to explicitly annontate such tasks
and scheduler modifications will be necessary to support this hybrid model.  As
large tasks have been shown to occupy most of the cluster's
resources~\cite{something}, we plan to explore the trade-off between flexibility
and utilization while support larger tasks.  While the projected benefits of
using tiny tasks (Section ~\ref{sec:benefits}) are realized when all the tasks
are uniformly small, we believe that supporting a few large tasks will still
provide significant benefits.

\eat{
- Major point: Data intensive applications have a lot of state -- and
this is hard to checkpoint. Need a crisp quantitative argument for
this.
- Lots of work in distributed OS about this - Amoeba, Sprite, Chorus
etc. Also VM migration work. Need to differentiate from them.
- Argue for global scheduling as we want load balancing across the datacenter.
}

\subsection{Alternate Designs}
Our proposal to use a co-operative multi-tasking scheme is in contrast to
preemption based schemes commonly used in operating systems. The benefits of 
preemption and short task length are well studied in operating
systems literature; for instance~\cite{sherman1972trace} shows that operating
system schedulers should ideally be preemptive, and should prevent jobs from
capturing the CPU for too long a time. However achieveing low overhead
task-switching is more expensive in the distributed setting. 

Previous work in process
migration~\cite{douglis1991transparent,milojivcic2000process} and virtual
machine migration~\cite{xen-vm-move} have looked at transparently moving
computation across machines. However these methods present overheads as
preempting a long running task involves checkpointing the task context and its
associated state. For data intensive applications, this could involve saving
gigabytes of data to disk or transferring it over the network in case the task
is to be resumed on another machine. Such large data transfers can put a
significant load on the data center network. Using co-operative multi-tasking in
the form of tiny tasks removes most of the requirement for saving or restoring
state.  In our design, the distributed ``scratch space''(Section~\ref{sec:prog})
is an example of state that is maintained across task executions, but since
programmers explicitly specify data that needs to be saved, we believe this will
be more efficient than automatic checkpointing of all the memory allocated by a
task. 

In a datacenter setting, tiny tasks are also better suited to provide fault
tolerance when compared to task preemption. Since each tiny task is assumed to a
deterministic unit of work, tasks can be executed in parallel during recovery.
In contrast, using preemption we would need to maintain redundant checkpoints
and fault recovery cannot be executed in parallel. 

The major advantage of choosing preemption would be that task execution times
can be precisely controlled. However we believe that by controlling task input
sizes, making modifications to programming frameworks like MapReduce or 
Spark and using task running time estimation~\cite{something}, we believe that
task execution times can be controlled while using co-operative multi-tasking. 

\eat{
\subsection{Low-Overhead Scheduling}
Breaking each task into hundreds or thousands of tiny tasks leads to 2-3 orders
of magnitude more scheduling decisions that the scheduler needs to make.
While the Hadoop job tracker (for example) has a well-known scalability limit
at approximately 4000 machines, recent work~\cite{sparrow} has shown that
distributed scheduilng provides a near-optimal and highly scalable alternative
to centralized schedulers. Might also want to cite YARN here, even though it
does not work as advertised.

Can use leases to further improve scalability.

Early frameworks (e.g., MapReduce) launch a
new JVM for each task, so launching a task takes seconds.  Earlier
work~\cite{Amoeba} has used high task launch overheads to argue against
using smaller task sizes.  Launching a new JVM for each task is
unnecessary and costly; newer frameworks like Spark don't do this and can
launch tasks in milliseconds (true?).

\subsection{Scalable File System}
In order for tasks to be short, they must operate on a small amount of on-disk
data, or use in-memory data.
For data stored on-disk, assuming data can be read from disk at 100MB/s,
a task can read no more than 10MB of data if it is to complete in 100ms.
Current file systems like HDFS already hit scalabilty limits in many clusters
using the default block size of 64MB (Facebook uses 1GB block sizes for this
reason), so could not handle the metadata needed to store small file blocks.
The recent Flat Data Center Storage~\cite{fds} work solved this problem
by distributing meta data in the magic table; tiny tasks would be built
atop a file syste like FDS.

To make tasks complete even faster, tasks may operate on in-memory data.
Frameworks like Spark and PacMan store disk in memory, allowing a task that
completes in 100ms to scan as much as 1GB of data.

Comment on memory: we anticipate needing some degree of locality, unlike FDS,
since there will always be a storage technology with higher bandwidth than
is current cost effective for networking. Scheduling in this environment
is an open problem but might rely on something like delay scheduling, which
can be implemented in distributed schedulers (see Sparrow).

\subsection{How Low Can We Go? (move this throughout!)}
Given the architecture described above and current technologies, we expect that
running tasks shorter than 100ms will lead to significant overheads. With disk
blocks lower than 8MB, can no longer amortize seek times, making disk reads take
longer. (use progress rate graphs to show what task lengths this would lead to?)

We are hopeful that storing data in memory and further
improving scheduling can drive task launch overheads to <1ms, making tasks that
complete in tens of milliseconds practical.

\subsection{What About Tasks that Cannot Easily Be Parallelized?}
As a starting point, we consider the improvements from ``tiny-ifying'' only the
tasks that are easily parallelized.  Figure~\ref{??} depicts the performance of
tiny tasks as we decrease the percentage of tasks that can be parallelized, based
on traces from Facebook.  There is an elbow at ??.

As shown in Figure~\ref{??}, using tiny tasks as a design principle improves
performance by X\%, but when tasks cannot be parallelized, we leave X\% of
performance on the table.  To enable making \emph{all} tasks tiny, we propose
creating a new computing framework that provides distributed ``scratch space.''
Parallelize by storing some stuff in this scratch space, so it can be accessed
by multiple tasks.  Give an example (e.g., unique). Provide some estimate
of how many tasks fall in this category?
}
